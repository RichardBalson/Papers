\section{Results}

In the results section the estimation of the Wendling model is considered. Initially, the model simulations used as observations for the estimation procedure are demonstrated. In particular, it is shown that the model is capable of replicating six different types of neural activity. Following this, the estimation of the model under numerous conditions is considered. 

\subsection{Model Simulation}

\red{Model simulation.}
The Wendling model is capable of replicating six types of neural activity often observed in intracranial EEG from the hippocampus of epileptic patients. The different types of activity are attained by altering the synaptic gains in equations~\ref{eqn: EulerW1}-\ref{eqn: EulerW8}. The results of the simulation are summarised in figure~\ref{fig: SimResults} and are similar to those published by~\cite{wendling2002epileptic}. 
\begin{figure}%%%%%%%%%%%%%%%%%%%%%%%%%%%%%%%%%%%%%%%%%%%%%%%%%%%%%%5
	\centering
		\includegraphics[width = 1\textwidth]{../../Images/Images/pdf/Wendling_Sim_Results_all.pdf}
	\caption{Wendling model simulation results. Each figure corresponds to a different excitatory synaptic gain ($\theta_{p}$). The x and y axes correspond to the fast ($\theta_{fi}$) and slow inhibitory ($\theta_{si}$) synaptic gains for each simulation. The labels, BG, SS, SDS, SR, LVR and QS correspond to background, sporadic spikes, sustained discharge of spikes, slow rhythmic, low voltage rapid and quasi-sinusoidal activity. This image is created by altering the synpatic gains of each populations. The resolution used is 0.5 for $\theta_{p}$ and 5 for $\theta_{si}$ and $\theta_{fi}$.}
	\label{fig: SimResults}
\end{figure}%%%%%%%%%%%%%%%%%%%%%%%%%%%%%%%%%%%%%%%%%%%%%%%%%
\red{Observations from simulation results}

From figure~\ref{fig: SimResults} it can be seen that as $\theta_{p}$ increases the number of transitions between types of neural activity grows. Further, it is noted that for low $\theta_{p}$ mostly background EEG is observed. In seizure activity there are numerous transitions in the type of activity observed. It would be expected that the slow states describing the recorded EEG would occur in regions of the state space where there are numerous transitions. These areas of the parameter space occur when $\theta_{p}$ is greater than four, $\theta_{si}$ is between five and thirty and $\theta_{fi}$ is less than twenty. This can be seen in figure~\ref{fig: SimResults} when $\theta_{p}$ is five. 

\subsection{Fast State Estimation}
\label{ssec: MEstimation}

\red{Estimation of states, with known parameters.}
Firstly, the estimation of the model's fast states is considered, where slow states are considered to be known. EEG is simulated and used as observations for the estimation procedure to demonstrate the UKF's ability to track the model's fast states. The simulated EEG is corrupted by an additive Gaussian noise source with a known variance. Figure~\ref{fig: EstFSO} shows the simulation and estimation results for the membrane potentials of each neural population.
\begin{figure}%%%%%%%%%%%%%%%%%%%%%%%%%%%%%%%%%%%%%%%%%%%%%%%%%%%
	\centering
		\includegraphics[width =1\textwidth]{../../Images/Images/pdf/Gauss_UKF_WM8_f=2048_A_5_B_20_G_20_DC_S_2_P_0PE_Gauss_N_1mV_Model_States_InputsZ.pdf}
	\caption{Results for the estimation and simulation of fast states. For the simulated signal the slow states are specified to occur at the midpoint of their physiological range. The estimation results are shown in red (Est.), and the simulated observations in black (Sim.). The membrane potentials shown in this result correspond with those demonstrated in figure~\ref{fig: Biological}, where $v_{p}$, $v_{e}$, $v_{si}$ and $v_{fi}$ correspond to the membrane potentials of the pyramidal, excitatory, slow inhibitory and fast inhibitory neural populations, respectively.}
	\label{fig: EstFSO}
\end{figure}%%%%%%%%%%%%%%%%%%%%%%%%%%%%%%%%%%%%%%%%%%%%%%%%%%%%%%%

\red{Observations from estimation of fast states} 

Figure~\ref{fig: EstFSO} shows that the UKF can track the fast states from the model. However, it is clear that the UKF is better at tracking the fast states slower dynamics. That being said the UKF is still capable of tracking fast dynamics in these states, but not as accurately. The reason for this is that the uncertainty in the filter at times cannot account for the stochastic input to the model. 

\subsection{Slow State Estimation}
\label{ssec: SSEstimation}

\red{Estimation of a slow state.}

Next the addition of a single slow state to the estimation procedure is considered. All estimation setting are identical as those used in the fast state estimation procedure with the addition of a slow state to the augmented state matrix. In figure~\ref{fig: ESTOSFP} the results for the estimation of the synaptic gain of the pyramidal and excitatory neural populations are demonstrated. 

\begin{figure}
	\centering
		\includegraphics{../../Images/Images/pdf/Gauss_UKF_WM8_f=2048_A_5_B_20_G_20_DC_S_2_P_1PE_Gauss_N_0mV_Excitation.pdf}
	\caption{Estimated and simulated excitatory synaptic gain ($\theta_{p}$). For this estimation the inhibitory slow states are considered to be known. Red and black indicate the estimated (Est.) and simulated (Sim.) excitatory synaptic gain, and green the error (Std. Dev.) on the estimate.}
	\label{fig: ESTOSFP}
\end{figure}

\red{Observations from estimation of single slow state}
Figure~\ref{fig: ESTOSFP} shows that the estimation of this slow state oscillates around its actual value. The variation in this estimation is due to the stochastic input. However, using the mean of the estimation procedure may allow for greater accuracy in the final estimate.

\red{Estimation of three parameters.}

Next the estimation of all slow states is considered. To achieve this the state matrix is augmented with all model slow states. The UKF settings are identical to those used for the estimation of a single slow state. In figure~\ref{fig: ESTMP3} the result for this estimation procedure is demonstrated.
\begin{figure}%%%%%%%%%%%%%%%%%%%%%%%%%%%%%%%%%%%%%%%%%%%%%%%
	\centering
		\includegraphics[width =1\textwidth]{../../Images/Images/pdf/Gauss_UKF_WM8_f=2048_A_5_B_20_G_20_DC_S_2_P_3PE_Gauss_N_1mV_Vsi.pdf}
	\caption{Estimated and simulated slow states. Here three slow states are considered: excitatory ($\theta_{p}$) and slow ($\theta_{si}$) and fast inhibitory ($\theta_{fi}$) synaptic gains. Red and black indicate the estimated (Est.) and simulated (Sim.) slow states, and green the error (Std. Dev.) on the estimate.}
	\label{fig: ESTMP3}
\end{figure}%%%%%%%%%%%%%%%%%%%%%%%%%%%%%%%%%%%%%%%%%%%%%%%%%

\red{Observations from estimation of three slow state}

Figure~\ref{fig: ESTMP3} demonstrates that all model slow states can be estimated. Further, it can be seen that the estimation of inhibitory slow states are more accurate in this procedure, and converge to their true values in a short time frame.

\subsection{Estimation Robustness}
\label{ssec: ESTT}

Up to this point it has been shown that the estimation procedure works under specific observation noise and state initialisation error. In the section the performance of the UKF under varying conditions is considered.

\subsubsection{Estimation with Varying Observation Noise}

\red{Estimation of three parameters with varying levels of observations noise.}

For the dual estimation problem, results described in section~\ref{ssec: SSEstimation} are obtained when observation noise is low; however, for iEEG recordings this will not necessarily be the case. Therefore, it is necessary to consider how robust the estimation procedure is to noise. To determine this the noise added to the simulated signal is increased. The results of this procedure are shown in figures~\ref{fig: ESTFail}-\ref{fig: ESTPass}.
\begin{figure}%%%%%%%%%%%%%%%%%%%%%%%%%%%%%%%%%%%%%%%%%%%%%%%%%%%%%%%%%%%%
	\centering
		\includegraphics[width =1\textwidth]{../../Images/Images/pdf/Gauss_UKF_WM8_f=2048_A_5_B_20_G_20_DC_S_2_P_3PE_Gauss_N_700mV_Model_Parameters.pdf}
	\caption{Estimation and simulation results for high observation noise. In this estimation procedure the noise added to the system has a standard deviation of 700mV. Three slow states are considered: excitatory ($\theta_{p}$) and slow ($\theta_{si}$) and fast inhibitory ($\theta_{fi}$) synaptic gains. Red and black indicate the estimated (Est.) and simulated (Sim.) slow states, and green the error (Std. Dev.) on the estimate.}
	\label{fig: ESTFail}
\end{figure}
\begin{figure}
	\centering
		\includegraphics[width =1\textwidth]{../../Images/Images/pdf/Gauss_UKF_WM8_f=2048_A_5_B_20_G_20_DC_S_2_P_3PE_Gauss_N_650mV_Model_Parameters.pdf}
	\caption{Estimation and simulation results for large observation noise. In this estimation procedure the noise added to the system has a standard deviation of 650mV. Three slow states are considered: excitatory ($\theta_{p}$) and slow ($\theta_{si}$) and fast inhibitory ($\theta_{fi}$) synaptic gains. Red and black indicate the estimated (Est.) and simulated slow states, and green the error (Std. Dev.) on the estimate.}
	\label{fig: ESTPass}
\end{figure}%%%%%%%%%%%%%%%%%%%%%%%%%%%%%%%%%%%%%%%%%%%%%%%%%%%%%%%%%%%%%%%%

\red{Observations from estimation with varying levels of noise}

In figure~\ref{fig: ESTPass} the observation noise level is set to 650mV and in figure~\ref{fig: ESTFail}, 700mV. These two figures show that an increase in observation noise can lead to the failure of the estimation procedure.

\subsubsection{Estimation with Varying Initialisation State Error}

\red{Estimation of parameters with different percentage error initial parameter guesses.}
When considering the estimation of states the distance between their actual and initialised values could result in incorrect convergence, due to local minima. To determine how robust the estimation procedure is to the distance between the actual and initialised states numerous estimations are performed with different initialisations. The results are shown in figure~\ref{fig: EstInitFail}
\begin{figure}%%%%%%%%%%%%%%%%%%%%%%%%%%%%%%%%%%%%%%%%%%%%%%%%%%%%%%%%%%%%
	\centering
		\includegraphics[width =1\textwidth]{../../Images/Images/pdf/Gauss_UKF_WM8_f=2048_A_6_B_21_G_32_DC_S_2_P_3PE_Gauss_N_10mV_Model_Parameters.pdf}
	\caption{Estimation and simulation results when slow states are initialised with a high percentage error from their actual values. Three slow states are considered: excitatory ($\theta_{p}$) and slow ($\theta_{si}$) and fast inhibitory ($\theta_{fi}$) synaptic gains. Red and black indicate the estimated (Est.) and simulated (Sim.) slow states, and green the error (Std. Dev.) on the estimate.}
	\label{fig: EstInitFail}
\end{figure}%%%%%%%%%%%%%%%%%%%%%%%%%%%%%%%%%%%%%%%%%%%%%%%%%%%%%%%

\red{Observations from estimation of parameters with different percentage error initial parameter guesses.}

Figure~\ref{fig: EstInitFail} demonstrates the importance of the initialisation of the model's slow states. If the slow states are initialised with a high level of inaccuracy the estimation procedure does not converge to the true slow state values.However the value the estimates converge to is physiologically implausible . Therefore, the estimates are considered to be an inaccurate description of the observed activity.

This problem can be solved by conditioning the results of the estimation procedure. The estimation algorithm determines whether the final estimate for each slow state occurs within its physiological range. If the final estimate occurs outside its physiological range then the initialised states are redrawn from a Gaussian distribution and the estimation procedure is repeated. In figure~\ref{fig: EstInitPass} an example of the estimation is shown, where states converge after a longer duration in time.
\begin{figure}%%%%%%%%%%%%%%%%%%%%%%%%%%%%%%%%%%%%%%%%%%%%%%%%%%%
	\centering
		\includegraphics[width =1\textwidth]{../../Images/Images/pdf/Gauss_UKF_WM8_f=2048_A_6_B_12_G_27_DC_S_2_P_3PE_Gauss_N_10mV_Model_Parameters.pdf}
	\caption{Estimation and simulation results when slow states are initialised with a lower percentage error than figure~\ref{fig: ESTFail} from their actual values and are condtioned such that final estimates need to be pysiologically plausible. Three slow states are considered: excitatory ($\theta_{p}$) and slow ($\theta_{si}$) and fast inhibitory ($\theta_{fi}$) synaptic gains. Red and black indicate the estimated (Est.) and simulated (Sim.) slow states, and green the error (Std. Dev.) on the estimate.}
	\label{fig: EstInitPass}
\end{figure}%%%%%%%%%%%%%%%%%%%%%%%%%%%%%%%%%%%%%%%%

\subsection{Estimation of Varying Parameters}
\red{Estimation of parameter with varying simulated parameters.}

All previous results have assumed that the simulated observations are stationary, at least in a sense that the slow state values used to simulate the model are constant throughout. However, the brain is not stationary, and it is the altering of slow states that results in the various seizure like activity that are observed in the simulated model (figure~\ref{fig: SeizureSim}). Therefore, it is necessary to consider the estimation of parameters when they are altering within a single simulation. Here the estimation of a simulated seizure and the corresponding membrane potentials of each neural population are shown in figure~\ref{fig: ESTMP3VPS}. Further, the estimated values for the slow states are demonstrated in figure~\ref{fig: ESTMP3VP}.

\begin{figure}
	\centering
		\includegraphics[width =1\textwidth]{../../Images/Images/pdf/Gauss_UKF_WM8_VP_f=2048_A_5_B_40_G_20_DC_S_2_P_3PE_Gauss_N_10mV_Model_States_Inputs1.pdf}
	\caption{Estimation and simulation of membrane potentials of each neural populations as the model transitions from background to seizure EEG. The membrane potentials shown in this result correspond with those demonstrated in figure~\ref{fig: Biological}, where $v_{p}$, $v_{e}$, $v_{si}$ and $v_{fi}$ correspond to the membrane potentials of the pyramidal, excitatory, slow inhibitory and fast inhibitory neural populations, respectively. Red here indicates the estimate (Est.) for the membrane potential and black its simulated (Sim) value.}
	\label{fig: ESTMP3VPS}
\end{figure}
\begin{figure}
	\centering
		\includegraphics[width =1\textwidth]{../../Images/Images/pdf/Gauss_UKF_WM8_VP_f=2048_A_5_B_40_G_20_DC_S_2_P_3PE_Gauss_N_10mV_Model_Input_and_Parameters.pdf}
	\caption{Estimation and simulation results for varying slow states used to generate the seizure in figure~\ref{fig: ESTMP3VPS}. Slow states are altered every five seconds, with a transition time. Three slow states are considered: excitatory ($\theta_{p}$) and slow ($\theta_{si}$) and fast inhibitory ($\theta_{fi}$) synaptic gains. Red and black indicate the estimated (Est.) and simulated slow states, and green the error (Std. Dev.) on the estimate.}
	\label{fig: ESTMP3VP}
\end{figure}

\red{Observations from estimation of seizure}

In figure~\ref{fig: ESTMP3VPS} the membrane potentials of each neural populations is shown. Here $v_{p}$ is the model output, and in this case a simulated seizure. There are four different parameter sets used to simulate this seizure. Each corresponds to a five second epoch in figure~\ref{fig: ESTMP3VPS}. Four types of neural activity can be seen, these being background (1-5s), interictal (6-10s), low amplitude high frequency (11-15s) and seizure (16-20s). The estimation procedure is capable of tracking the model's membrane potentials (figure~\ref{fig: ESTMP3VPS}). In figure~\ref{fig: ESTMP3VP} the estimation of the slow states is shown. This figure shows that the estimation procedure is capable of tracking slow states when they vary within a single simulation.  

\subsection{Estimation of Model Input}

In epileptic patients it is unreasonable to assume that the mean input firing rate from other regions of the brain remains constant. This is particularly true during seizure periods. Therefore, the estimation procedure is further extended to the case where the input mean is considered a slow state. The results from this estimation procedure are shown in figure~\ref{fig: ESTMP4}.

\begin{figure}
	\centering
		\includegraphics[width =1\textwidth]{../../Images/Images/pdf/Gauss_UKF_WM8_f=2048_A_5_B_20_G_20_DC_S_2_P_4PE_Gauss_N_1mV_Model_Input_and_Parameters.pdf}
	\caption{Estimation and simulations results for all model slow states including the input mean. Four slow states are considered: excitatory ($\theta_{p}$) and slow ($\theta_{si}$) and fast inhibitory ($\theta_{fi}$) synaptic gains as well as the input mean. Red and black indicate the estimated (Est.) and simulated (Sim.) slow states, and green the error (Std. Dev.) on the estimate.}
	\label{fig: ESTMP4}
\end{figure}

\red{Observations from estimation of four slow state}

Figure~\ref{fig: ESTMP4} shows that the input mean can be estimated, while estimating the model's slow states. This is an important results, as it shows that it is theoretically possible to estimate the effect of afferent neural masses on the modeled neural population.

\red{Estimation of all model parameters while varying parameters in a single simulation}
Lastly, consider the case that seizure activity at the focus results in a spread of a seizure. This spread of seizure activity would alter the normal activity of the neural populations connected to the modeled region. This abnormal activity would result in variations in the mean of the input to the modeled population. Therefore, the situation where slow states are varying is considered again. But in this case it is assumed that the input mean is unknown, and it is estimated. The results for his estimation procedure are shown in figure~\ref{fig: ESTMP4VP}.

\begin{figure}
	\centering
		\includegraphics[width =1\textwidth]{../../Images/Images/pdf/Gauss_UKF_WM8_VP_f=2048_A_5_B_40_G_20_DC_S_2_P_4PE_Gauss_N_10mV_Model_Input_and_Parameters.pdf}
	\caption{Estimation and simulations results for all varying model slow states, used to generate seizure EEG, as well the input mean. Four slow states are considered: excitatory ($\theta_{p}$) and slow ($\theta_{si}$) and fast inhibitory ($\theta_{fi}$) synaptic gains as well as the input mean. Red and black indicate the estimated (Est.) and simulated (Sim.) slow states, and green the error (Std. Dev.) on the estimate.}
	\label{fig: ESTMP4VP}
\end{figure}

\red{Observations from estimation of parameters and input for simulated seizure}
Figure~\ref{fig: ESTMP4VP} shows that the estimation procedure is capable of tracking the model parameters and estimating the input mean at the same time. It is worth noting that when the estimation of the input mean varies from its actual value the estimation of both the excitatory ($\theta_{p}$) and fast inhibitory synaptic gains ($\theta_{fi}$) do not vary much when compared to slow inhibition ($\theta_{si}$). However, the synaptic gain of the slow inhibitory ($\theta_{si}$) population varies from its actual value when the estimation of the input mean is incorrect. This suggests that these two parameters are inherently linked. 


\red{Estimation of single parameter set with different simulated signals due to stochastic input}

\subsection{Multiple Simulated Signal Estimation}
\label{ssec: MSSE}

All of the estimation's procedure discussed above have considered different simulated model outputs. This is in light of the fact the the input to the model is stochastic in nature. Therefore, it is necessary to determine the performance of this estimation procedure over numerous simulated observations. To test the estimation procedure, slow states are held constant for the simulated observations; however, the input is altered for each simulation. This is repeated one hundred times. For each of these simulated signals the slow states and input mean are estimated. In figure~\ref{fig: MultiSim} the mean and standard deviation for each model slow state is shown. 

\red{Observations of multiple simultion results.}
The results in figure~\ref{fig: MultiSim} show that the UKF can accurately estimate all model slow states with multiple different stochastic inputs. All results show that the true value for each slow state is within one standard deviation of its estimated value.

\begin{figure}[ht]
\centering
\subfloat[]{	\includegraphics[scale=0.43]{pdf/Estimation_Results_Multiple_Simulations_Gaussian_DistExcitatory_Gain.pdf}
}
\subfloat[]{	\includegraphics[scale=0.43]{pdf/Estimation_Results_Multiple_Simulations_Gaussian_DistSlow_Inhibitory_Gain.pdf}
}\\
\subfloat[]{	\includegraphics[scale=0.43]{pdf/Estimation_Results_Multiple_Simulations_Gaussian_DistFast_Inhibitory_Gain.pdf}
}
\subfloat[]{	\includegraphics[scale=0.43]{pdf/Estimation_Results_Multiple_Simulations_Gaussian_DistInput_mean.pdf}
}
\caption{Estimation results for 100 different simulations with identical slow states. For this estimation procedure slow states are constant throughout each simulation. Their values re indicated by the red vertical line. The standard deviation of the resulting estimation procedure is demonstrated by the blue vertical lines.}
\label{fig: MultiSim}
\end{figure}

