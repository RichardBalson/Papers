\begin{abstract}

%\red{ What is it about}

In this paper, a method to estimate a neural mass model of the hippocampus~\citep{wendling2002epileptic} is discussed. The model is estimated using an unscented Kalman filter. Estimating the model using this filter reduces computation time, while at the same time retaining accuracy. This reduction in computation time makes this technique viable for imaging aspects of the brain that are not directly observable. Here it is shown that the unscented Kalman filter can track the model parameters from the neural mass model of the hippocampus under various conditions. This will allow imaging the effect that neurological treatments have on the brain, and possibly allow for new medical treatments to be developed, such as new stimulation strategies to reduce seizure frequency in epilepsy patients. Further, this technique may allow for more advancement in seizure anticipation.

%\red{Why do it?}
%
%\red{How you did it?}
%
%\red{Results and what it means}

%A neural mass model described by \cite{wendling2002epileptic} has been shown to be capable of mimicking iEEG activity from the hippocampus. For this to occur gain parameters in the model are altered which mimick different types of iEEG. However, the estimation of this model from recorded iEEG has only been shown for partially stationery epochs. This method allows for a general understanding of how parameters alter pre- and post-seizure; however, it does not fully characterise the changes occurring. In this study, a method to determine model parameters based on each recording made will be implemented. This involves the implementation of an unscented Kalman filter (UKF). In this paper, the implementation and the results of the UKF on data simulated using a neural mass model are shown. In particular, it is shown that the UKF is capable of tracking changes in simulated data model parameters accurately under various conditions. 



\end{abstract}