\section{Introduction}

\red{Aims}

\red{Estimate model parameters from a neural mass model of the hippocampus.}
Epilepsy is not well understood. At present, the mechanism underlying the generation of seizures are unknown, and approximately one third of patients with epilepsy are refractory to treatment. In this paper, a new method to estimate physiological aspects of the brain is introduced. This method involves the application of a well recognised neural mass model with a relatively new estimation technique. In particular, the application of an unscented Kalman filter (UKF)~\citep{voss2004nonlinear} for estimation of physiologically relevant parameters from a neural mass model of the hippocampus~\citep{wendling2002epileptic} (refered to as the ``Wendling Model'') using EEG recordings is considered. It is hypothesised that, using the UKF, parameters from the Wendling model can be tracked with noisy observations.

\red{To improve the understanding of epilepsy and improve the treatment of epilepsy patients.}
By estimating the physiologically relevant parameters from the Wendling model, it will be possible to approximate physiological changes occurring in the brain that lead to seizure. This will be achieved by estimating model parameters based on electrographic recordings of seizures from the hippocampus. Further, this method can be applied to determine the effect that treatment has on the model parameters. Approximating the brain's physiology in this manner will make it possible to titrate therapies that are patient-specific and optimise their efficacy. For example, if it is observed that the model parameter describing inhibition decreases prior to seizure, a treatment that has the opposite effect can be determined and used for the specific patient. By doing so the therapeutic benefit from treatment for individual patients may be improved.

\red{How has this been achieved previously?}

\red{Introduction to neural mass model, freeman, jansen etc}
Neural mass models, originally formulated in the early 1970's~\citep{wilson1973mathematical,lopes1974model,freeman1963electrical} and developed subsequently~\citep{jansen1995electroencephalogram}, describe a cortical region of the brain as having populations of inhibitory and excitatory neurons. The neurons are modeled by two function. The first describes the action of the dendrites in terms of a delay and a synaptic gain. The delay specifies the time taken for action potentials to propagate from one population to the next, and the synaptic gain is an indication of the membrane potential magnitude resulting from a single action potential arriving at the considered population. The second function describes how the membrane potential of each neural population is converted into a firing rate. Here firing rate indicates the aggregate number of action potentials generated from the considered population. The second function is modeled as a sigmoid, which was originally formulated to describe the probability of a neuron firing given a specific membrane potential. In this case, population dynamics are considered. Therefore, the sigmoid results in an aggregate firing rate given a specific population membrane potential. Lastly, populations are connected together and the firing rate from each population is scaled based on the number of synaptic connections between the populations. The excitatory neurons primarily consist of pyramidal neurons which are known to have similar orientations. The parallel orientation of the pyramidal neurons allows for the electrical fields that they produce to aggregate together. For other neural populations, orientation is random and, therefore, their net effect on the measured electric field is minimal. Therefore, in this model, the pyramidal neurons are considered to be the generators of observed EEG. One such neural mass model is capable of replicating normal, or background, EEG as well as slow rhythmic activity or alpha waves by altering model parameters~\citep{jansen1995electroencephalogram}.

\red{Inadequacy of jansen model and intro to the wendling model}
The model described by \cite{jansen1995electroencephalogram} was shown to be capable of replicating other key features observed in EEG, such as quasi-sinusoidal seizure electrographic activity. However, the model was not able to replicate a key feature observed prior to seizure in recordings from the hippocampus, namely low amplitude, high frequency oscillations. A study performed by \cite{white2000networks} showed that within the hippocampus the effect of inhbition on the pyramidal population had two distinctly different propagation delays, and that both were significant for the reproduction of EEG. They hypothesised that the cause of the two different propagation delays were due to the location of the synapses connecting the inhibitory and pyramidal populations. Longer propagation delays are due to synapse connections far from the soma (peri-dendritic), and shorter delays are due to connections near the soma (peri-somatic).  This effect was incorporated into the neural mass model by \cite{wendling2002epileptic}. In order to account for the two propagation delays, the Wendling group described two different types of inhibitory populations: one fast (peri-somatic), and the other slow (peri-dendritic) inhibitory. In the same study, it was shown that the addition of the peri-somatic inhibitory population made it possible for the neural mass model to replicate the key characteristics of low amplitude, high frequency oscillations.

\red{This model is capable of replicating key characteristics observed in EEG prior to and during seizure.}

The Wendling model is capable of replicating key features observed in EEG prior to, during, and after seizures. This is achieved by altering physiological parameters that describe the balance between excitation and inhibition in the modeled region of the brain. Due to its description of neuronal connections and systems in terms of neural populations, the model only has ten parameters. \cite{wendling2002epileptic}, demonstrated that key features observed prior to, during and after seizures can all be replicated by altering three of these model parameters, which describe the balance between excitation and inhibition. Therefore, to mimic the observed output of iEEG, it is necessary to be able to estimate these three model parameters.
\red{Previous work on estimating the neural mass model of the hippocampus has been done using a genetic algorithm.}

The neural mass model of the hippocampus has previously been estimated using the genetic algorithm~\citep{wendling2005interictal}. The genetic algorithm is capable of estimating model parameters iteratively. The iteration ensures that the genetic algorithm converges, although it may converge to the incorrect parameters. For the the genetic algorithm to converge the data analysed needs to be stationary, i.e. model parameters need to be constant over the considered period of observations. The advantage of the UKF over this method is that there is no requirement that the observations are stationary, as it can track the changes in model parameters. This is important as subtle changes in model parameters may give an indication of when a seizure is about to occur, and could provide evidence of the effect of therapeutic treatments on the brain's physiology. Further, the UKF can track model parameters in real time~\iref, which may allow the algorithm to be implemented in applications such as responsive stimulation and seizure prediction.

%The genetic algorithm is accurate; however, it is very time consuming due to its computational complexity. Genetic algorithm deterministic, cant account for stochastisity

%Due to the computation time this technique cannot be used over large data sets

\red{What is being done, and why is it better or different?}


In this paper, the application of the UKF for the estimation of the three model parameters describing the balance between excitation and inhibition is considered. Initially, artificial EEG is generated using the Wendling model, which is then used as the observations for the estimation procedure. Model parameters are then estimated, under the assumption that they were originally unknown. The robustness of the estimation procedure is determined by evaluating the accuracy of estimation under conditions where observation noise is varied and states are initialised with different percentages of error from their actual values.

\red{Why the Kalman filter}
The UKF, unlike the genetic algorithm, does not rely on iteration and is, therefore, less time consuming. This comes at the cost of accuracy. This paper looks at the accuracy of the filter under numerous conditions to determine how robust it is. If the UKF is accurate at tracking model parameters then this method could be used to help characterise full EEG data sets, and allow for treatments to be evaluated and developed. For example, if the UKF shows that inhibition decreases during seizure for a specific patient, a treatment that increases inhibition can be determined and used. This method may allow for patient specific treatments to be developed.

\red{Structure of the paper}
In the methods section, a description of the neural mass model of the hippocampus is presented, as well as the equations used to simulate the model. Further, the formulation of the UKF for the Wendling model is described. In the results section, the performance of the algorithm under numerous conditions are demonstrated. Lastly, in the discussion section an evaluation of the performance of the filter is provided, discussing whether this method is a viable way forward to use model estimation to help approximate the effect that disorders and treatments have on the brain. 