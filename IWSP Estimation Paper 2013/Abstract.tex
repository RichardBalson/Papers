\section{Abstract}

Epilepsy is a an unpredictable, and debilitating disorder that affects approximately 1\% of the world's populace. To date, methods attempting to predict seizures and evaluate epilepsy therapies have been unsuccessful. In this paper, we discuss a model based approach to provide further insights into physiological changes occuring in the brain prior to and during seizure. Further, we propose using such a model based approach to evaluate the efficacy of epilepsy therapies. In particular, we show that neural mass models can be approximated using an unscented Kalman filter, and that this procedure can be used to observe physiological changes in observed EEG, that are not elucidated by standard EEG evaluation technqiues. We propose that such a method can be used to evaluate therapies: first by characterising seizures in terms of observed model changes in neural physiology, and then by determing the effect of therapy. We show that efficacious therapies do alter estimated physiology in a manner that is predicatable (in the sense that approximated physiolgy is altered in a manner that is dissimilar to the changes observed during seizure). 

Preliminary results demonstrate that such a model based approach can indeed provide insight into neural dynamics during seizure. Further, using an unscented Kalman filter this procedure can be achieved in real time, allowing for quick evaluation of therapies, and more patient specific treatments.