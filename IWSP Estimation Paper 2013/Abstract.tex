\section{Abstract}

Epilepsy is an unpredictable, and debilitating disorder that affects approximately 1\% of the world's populace. To date, the mechanisms behind the generation of seizures are not amply understood. In this paper, we discuss a model based approach to provide further insights into physiological changes occurring in the brain prior to, and during seizure. Further, we propose using such a model based approach to evaluate the efficacy of epilepsy therapies. In particular, we show that neural mass models can be approximated using an unscented Kalman filter, and that this procedure can be used to observe physiological changes in observed EEG, that are not elucidated by standard EEG evaluation techniqiues. We show that efficacious therapies do alter estimated physiology, and that this change in dynamics may allow for quick evaluation of different therapies. We propose that a model based approach can be used to evaluate the efficacy of different therapies: first by characterising seizures in terms of observed model changes in neural physiology, and then by determining the effect of therapy on neural dynamics.
%This is clear from the inconsistent results found with current seizure prediction techniques.
% in a manner that is predicatable (in the sense that approximated physiolgy is altered in a manner that is dissimilar to the changes observed during seizure). 

Preliminary results demonstrate that such a model based approach can provide insight into neural dynamics during seizure. Further, using an unscented Kalman filter, this procedure can be achieved in real time, allowing for quick evaluation of therapies, and more patient specific treatments.