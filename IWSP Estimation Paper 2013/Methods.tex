\section{Methods}

The evaluation of therapies for epilepsy requires a method that can provide insights into the neurological changes that occur during seizure. Neural field models provide such insight, and are capable of mimicking neural activity observed in EEG. The change in brain dynamics is often approximated using neural field models. The approximations occur over regions when the brain is transitioning from normal to seizure activity. The results from these studies are often in line with physiological experiments, and may indicate that these models can in fact be used to approximate the changes in the brain. 

Neural field models have parameters that are indicative of brain physiology. Neural field parameters need to be estimated using the electrographic recording from the brain, to gain insight into the changes in brain physiology during a seizure. This is often achieved by a method known as the genetic algorithm. The genetic algorithm is an estimation technique often used in non-linear problems. However, this method requires numerous iterations of the same procedure and can be time consuming and computationally expensive. Further, and most importantly, this method does not account for the stochastic nature of the brain. Another method that has recently been considered is the development of an observer function for particular parameters. However, for this study we consider a method that is both computationally efficient and can account for the stochastic nature of the brain. 

The method considered here is the unscented Kalman filter. The unscented Kalman filter is an extension of the Kalman filter. The addition to the Kalman filter is that of the unscented transform, which is used to approximate the nonlinearity in a model. The unscented transform makes the assumption that all model states and parameters are normally distributed. Further, it is assumed that the nonlinearity in the model can be adequately reconstructed by propagating the mean and points one standard deviation from the mean through the model. It is then assumed that the resultant states are normally distributed. This procedure is then used to determine a new expectation of each state as well as the uncertainty of each state. This procedure is demonstrated in figure~\red{figure}.

The unscented Kalman filter is often used in meteorology, to predict the weather. Similar to the brain, in meteorology not all relevant causal effectors are observed. However, due to the robustness of the unscented Kalman filter this is overcome by increasing the uncertainty of each state. This is similar to the addition of unknown noise is a standard model description.

Formally, the unscented Kalman filter can be written in a succient set of equations\iref. However, for this paper it is sufficient to know that the unscented Kalman filter involves two steps: prediction and correction. In the prediction step, model states are propagated through the model using the unscented transform, and are reformulated into an predicted output with known standard deviation. The correction step is the same for the standard linear Kalman filter\iref.

With an estimation method in hand, a model is required. Numerous neural field models have been formulated. Two of which are briefly discussed. The first neural mass model describes the interaction between two neural populations: excitatory and inhibitory\iref. The second describes the interaction of three neural populations\iref. Both of these models are capable of mimicking EEG observed during seizure\iref. For this study the estimation of both models is considered. The Jansen model is a model developed for modeling any neural population in the brain, whereas the Wending model was developed for studies involving the hippocampus. To demonstrate the efficacy of the unscented Kalman filter for the estimation of neural field models, the Wendling model is considered. This model is theoretically harder to estimate, due to the larger number of states and estimated parameters in the model.

