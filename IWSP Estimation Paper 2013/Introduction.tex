\section{Introduction}

Epilepsy is a highly debilitating disease, that affects approximately 1\% of the worlds populace. At present, about a third of people with epilepsy are refractory to drug therapies. In this study, we look at developing new methods to evaluate the efficacy of different epilepsy treatments. In particular, we demonstrate the neural mass models can be accurately estimated in real time. To demonstrate this, we show that simulated EEG, using a neural mass model, can be accurately estimated. With a real time method to evaluate the effect of therapies on the brains physiology it may become possible to determine the effect of therapies on neural physiology, which may further inform the decison with regard to which treatment option is the most suitable for a particular patient.

Studies have been conducted where a neural mass model is estimated. However, these studies usually make use of a genetic algorithm which is computationally expensive. Further, the genetic algorithm cannot completely account for the stochastic nature of the brain. Further, these studies assume that the mean of the input remains constant, which may not be the case. In this study, an unscented Kalman filter is applied to estimate model parameters. The unscented Kalman filter is a maximum likelihood estimator; therefore, the estimator determines the most likely states and parameters for each observation made. The resulting estimate at each time period depends on the previous state. Therefore, the unscented Kalman filter is a Markov process.

In this study, we apply an unscented Kalman filter to a neural mass model\iref. This particular model is derived from a course description of hippocampal physiology. This course description describes the action of neural populations (regions of pyramidal and interneurons), in terms of how they connect to each other. Further, each population is ascribed a specific time constant and synaptic gain. In this study, we consider the estimation of the three neural populations synaptic gains. The Wendling group have shown that by altering these three parameters, the model can replicate most phenomena observed in hippocampal EEG.