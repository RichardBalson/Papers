\section{Results}

The unscented Kalman filter results in estimates with a certain level of uncertainty. Therefore, incorrect state and parameter estimates should still be within the bounds of uncertainty. Initially, we demonstrate that this is indeed the case for both state and parameter estimates (Figure~\ref{fig: States}).


The results from the parameter estimate quite clearly deomstrate the efficacy of the unscented Kalman filter for this particular problem. However, due to the dynamics of the states it is hard to evaluate the efficacy of the algorithm for states. There, a zoomed in version of the results in figure~\ref{} are zoomed in such that the efficacy of the unscented Kalman filter can be evaluated (Figure~\ref{}).

It is clear from these results that the unscented Kalman filter can estimate model parameters when they are stationary. However, how well the algorithm performs when parameters are varying, as expected in real brain physiology, is still to be determined. In the next experiment, parameters are altered in order to simulate a seizure start and end (Figure~\ref{}). The parameters from this simulation are then estimated (Figure~\ref{}). The result s above demonstrate the efficacy of the algorithm for a single parameter initialisation, and noise input. Therefore, we next consider how well the algorithm performs under numerous initial conditions and noise inputs. Twenty EEG signals are simulated and then estimated (Figure~\ref{}). 

LAstly, we consider the addition of parameter, the input mean firing rate to the modeled population. The addition is necessary to determine whether the modeled region, at least according to the algorithm is completely responsible for seizure, or it is involve in a network process and may initiate or process seizure EEG. The results for multiple estimations of the four model parameters, input mean and three synaptic gains, are show in figure~\ref{}. In figure~\ref{}, we shown the results for one such estimation demonstrating the uncertainty of each parameter. For the input mean, this uncertainty contains the information regarding the variance of the specified input, and can be used as a further parameter of interest.