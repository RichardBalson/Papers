\section{Abstract}

Epilepsy is an unpredictable, and debilitating disorder that affects approximately 1\% of the world's populace. To date, the mechanisms behind the generation of seizures are not amply understood. In this paper, we discuss a model based approach to provide further insights into physiological changes occurring in the brain prior to, and during seizure. In particular, we show that a neural mass model can be approximated using an unscented Kalman filter, and that this procedure can be used to observe physiological changes in observed EEG, that are not elucidated by standard EEG evaluation techniques. To demonstrate this method, we have made use of an \textsl{in vivo} model of focal temporal lobe epilepsy, where tetanus toxin is injected into the rat hippocampus. Two depth electrodes are inserted into the rat hippocampus and used to record local field potentials, which are used as the observations for the estimation procedure. We make use of a neural mass model originally proposed by the Wendling group that has been shown to be a good phenomological model of hippocampal EEG. 

Preliminary results - from the estimation of 10 seizures from two different animals- demonstrate that the transition from background to seizure is due to a decrease in the excitatory and slow inhibitory synaptic gain. Further, it is observed that the fast inhibitory synaptic gain increases at the transition to seizure. The preliminary results also demonstrate that the recovery from seizure is due to an increase in the excitatory synaptic gain, while both inhibitory synaptic gains remain constant. 

A further addition to this study is the estimation of the input mean firing rate to the modeled neural mass. We add this model parameter to the unscented Kalman filter as an augmented state (similar to the estimated synaptic gains). The estimation results show that the mean of the input increases at seizure onset, and remains constant post ictal.

From the preliminary estimation results, we believe that in this \textsl{in vivo} model of focal temporal lobe epilepsy seizure could be due to a failure in both excitatory and slow inhibitory mechanisms, while peri-somatic, or fast inhibitory populations are hyper active. These results also tend themselves to the idea that seizure is a network phenomenon. It is plausible that both excitatory and slow inhibitory mechanisms fail often, and seizure does not occur, but that with the addition of increased network activity the focal region may initiate a seizure. However, this is still to be confirmed, and will be considered in future studies.
